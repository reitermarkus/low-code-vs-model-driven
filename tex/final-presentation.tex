\documentclass[aspectratio=169]{beamer}

\usetheme{metropolis}

\usefonttheme{professionalfonts}
\usepackage[familydefault,light]{Chivo}

\usepackage{lmodern}
\usepackage[english]{babel}

\usepackage{fontspec}
\defaultfontfeatures{Ligatures=TeX}

\usepackage{multicol}

\usepackage{listings}

\usepackage{datetime}
\setdefaultdate{\usdate}

\usepackage{graphicx}
\graphicspath{{assets/}}

\newcommand{\german}[1]{{#1}}

\title{Low-Code vs.\texorpdfstring{\\}{} Model-Driven Architechture}
\author{Markus Reiter}


\date{\formatdate{12}{01}{2021}}

\institute{supervised by Prof. Dr. Ruth Breu}

\begin{document}
  \maketitle

  \begin{frame}{Outline}
    \begin{itemize}
      \item Motivation
      \item Model-Driven Architecture
      \item Low-Code Architecture
      \item Criticisms
      \item Planned Project Procedure
    \end{itemize}
  \end{frame}

  \begin{frame}{Motivation}
    \begin{itemize}
      \item What can and can't low-code and model-driven tools do?
      \item Are they a viable alternative to traditional development?
      \item When to choose one approach over the other?
    \end{itemize}
  \end{frame}

  \begin{frame}{Model-Driven Architecture}
    \begin{itemize}
      \item provides a set of guidelines for the structuring of specifications
      \item code (fully or partially) generated from models, e.g. from UML diagrams
      \item aimed at developers to speed up development
    \end{itemize}
  \end{frame}

  \begin{frame}{Low-Code Architecture}
    \begin{itemize}
      \item provides pre-built application components
      \item graphical user interface for creating both the
            application logic as well as the user interface
      \item typically aimed at end-users rather than developers
    \end{itemize}
  \end{frame}

  \begin{frame}{Criticisms}
    \begin{itemize}
      \item Low-Code Architecture
        \begin{itemize}
          \item Unsuitable for implementing scalable and mission-critical applications.
          \item Increase in unsupported applications built by shadow IT,
                i.e. applications which are not controlled by a company's IT department.
        \end{itemize}
      \item Model-Driven Architecture
        \begin{itemize}
          \item UML diagrams lack details included in the code itself.
          \item “the Code is the design” - Should models be derived from code instead of code from models?
        \end{itemize}
      \item Do these approaches actually make development easier and cheaper?
    \end{itemize}
  \end{frame}

  \begin{frame}{Evaluation of Low-Code Tools}
    \begin{itemize}
      \item Find low-code tools in the following categories:
          \begin{itemize}
            \item open-source
            \item developed by well-known company
            \item developed by unknown company
            \item old/well-established platform
            \item new/unestablished
          \end{itemize}
      \item Set up each tool
      \item Build test application with each tool
    \end{itemize}
  \end{frame}

  \begin{frame}{Open Standard Business Platform (OBSP)}
    \begin{itemize}
      \item open-source
      \item plug-in for the Eclipse IDE
      \item community version of the commercial OS.bee product developed by COMPEX
      \item latest version over one year old
      \item does not work with latest version of the Eclipse IDE
    \end{itemize}
  \end{frame}

  \begin{frame}{Corteza Low Code}
    \begin{itemize}
      \item open-source
      \item part of the Corteza Project initiated by Crust Technology
      \item the Corteza Project includes a CRM solution built on top of Corteza~Low~Code, among other things
      \item web-based platform
      \item test by signing up with a GitHub or Google account or \\
            by deploying it locally using Docker
    \end{itemize}
  \end{frame}

  \begin{frame}{Oracle APEX (Application Express)}
    \begin{itemize}
      \item commercial
      \item initially released as Oracle Flows in 2000
      \item web-based platform
      \item test by signing up for an Oracle Cloud account or \\
            by requesting an APEX workspace
    \end{itemize}
  \end{frame}

  \begin{frame}{Simplifier}
    \begin{itemize}
      \item commercial
      \item initially released by iTiZZiMO in 2012
      \item web-based platform
      \item test by using the Simplifier Playground (data is wiped every day) or \\
            by requesting a Simplifier test instance
    \end{itemize}
  \end{frame}

  \begin{frame}{Mendix}
    \begin{itemize}
      \item commercial
      \item founded in 2005 as a subsidiary of Siemens
      \item web-based platform (Mendix Studio) and Windows application \\
            (Mendix Studio Pro) with advanced features
      \item test by signing up for a regular account which allows hosting unlimited applications (with 1GB of memory and 0.5GB of storage per application)
    \end{itemize}
  \end{frame}
\end{document}
