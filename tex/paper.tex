\documentclass[runningheads]{llncs}

\linespread{1.1}

\raggedbottom

\usepackage{float}

\usepackage{lmodern}
\usepackage[english]{babel}

\usepackage{fontspec}
\defaultfontfeatures{Ligatures=TeX}

\usepackage{multicol}

\usepackage{listings}

\usepackage{datetime}
\setdefaultdate{\usdate}

\usepackage{graphicx}
\graphicspath{{assets/}}

\newcommand{\german}[1]{{#1}}

\title{Low-Code vs.\texorpdfstring{\\}{} Model-Driven Architechture}
\author{Markus Reiter}


\PassOptionsToPackage{hyphens}{url}
\usepackage{hyperref}
\usepackage{cleveref}
\usepackage{nameref}

\addto\extrasenglish{
  \def\sectionautorefname{Section}
}

\usepackage[
  style   = numeric,
  sorting = none,
]{biblatex}
\bibliography{paper}

\begin{document}

\institute{University of Innsbruck, Austria}

\maketitle

\vspace{8em}

\begin{abstract}

In this paper we present a survey conducted by comparing low-code tools with model-driven tools for developing software in order to get a better understanding of the advantages and drawbacks of both approaches and whether they are a viable alternative to traditional software development.

\keywords{Low-Code \and Model-Driven \and Development}

\end{abstract}

\newpage

\section{Introduction}
\label{sec:introduction}

In this paper, we compare Low-Code Platforms to Model-Driven Architecture and try to determine when and where to use them over Model-Driven Architecture or vice-versa.

In order to make clear to the reader the difference between the two approaches we firstly provide definitions and characteristics of each paradigm in the following sections.

\subsection{What is Model-Driven Architecture?}
\label{ssec:what_is_model_driven}

With Model-Driven Architecture or Model-Driven Engineering, the goal is to standardise on models in a given domain. This is done in order to reduce code duplication and speed up development. Especially when bootstrapping a new project, it is helpful keep boilerplate code to a minimum. In addition to reducing development time, Model-Driven Engineering also tries to increase productivity by maximising compatibility between systems by using the aforementioned standardised models. \cite{wiki:model_driven_engineering} Generally, Model-Driven Architectures are geared towards developers, i.e. people who also have a good understanding of the underlying programming language(s) of the respective architecture. One common way Model-Driven Engineering is used in practice is code generation. Using code generation can speed up many processes: For example, given an API specification (e.g. using the industry standard OpenAPI \cite{openapis}), one can generate all code that is necessary for a client for this API. Swagger \cite{swagger} is one such tool which can generate API clients for many different programming languages. There are also tools which can generate a complete class hierarchy from an UML diagram; many IDEs offer such functionality either natively or via plug-ins. These are only a few examples of what falls into the category of Model-Driven Architecture, the main point is that all of them help developers reach their goal faster than doing the same tasks manually.

\subsection{What is a Low-Code Platform?}
\label{ssec:what_is_a_low_code_platform}

A low-code platform, as the name implies, requires a minimal amount of code to be written. Similarly to conventional programming (i.e. using a programming language), an IDE is used for creating programs using low-code platforms. Often, the IDE is part of the low-code platform itself instead of a standalone program as is the case for conventional programming. In most cases, the IDE of the platform and the low-code platform cannot be used separately so the term low-code platform is also commonly used to refer to the IDE interface itself. Low-code platforms are targeted towards users with little to no programming experience, which means the actual programming of a low-code system is done mainly via visual building blocks. These building blocks contain pre-built templates and functionality and can be combined to form a coherent user interface. Oftentimes, custom functionality can be implemented using a click-and-drag pattern between building blocks and by assigning an action accordingly. The available building blocks vary by platform which makes them less portable than a system written using a programming language. This is one common point of critique on low-code platforms.

\section{Evaluation of Low-Code Tools}
\label{sec:evaluation_of_low_code_tools}

\section{Evaluation of Model-Driven Tools}
\label{sec:evaluation_of_model_driven_tools}

\section{Findings}
\label{sec:findings}

\section{Conclusion}
\label{sec:conclusion}

\newpage
\printbibliography

\end{document}
